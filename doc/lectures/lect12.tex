\section{Лекция 12 (02.12)}
\subsection{Метод конечных элементов}
\subsubsection{Четырёхугольный элемент. Билинейный двумерный базис}
\subsubsection{Численное интегрирование внутри конечного элемента}
\subsection{Разбор программной реализации МКЭ}
\clisting{open}{"test/poisson_fem_solve_test.cpp"}
Численное решение уравнения Пуассона с граничными
условиями первого рода реализовано в файле
\ename{poisson_fem_solve_test.cpp}.
Будем рассматривать решение двумерной задачи на треугольниках (тест \ename{[poisson2-fem-tri]}).
В этом тесте определяется двумерная аналитическая функция
$$
f(x, y) = \cos(10 x^2) \sin(10 y) + \sin(10 x^2) \cos(10 x),
$$
и формулируется уравнение Пуассона с граничными условиями первого рода, для которого эта функция является точным решением.
Далее уравнение Пуассона решается численно и полученный численный результат сравнивается
с точным ответом. Норма полученной ошибки печатается в консоль.

В функции верхнего уровня происходит чтение неструктурированной
сетки, создание рабочего объекта, вызов решения с возвращением нормы полученной ошибки
и вывод данныех: сохранение двумерного поля решения в vtk-файл и печать нормы в консоль:
\clisting{pass}{"[poisson2-fem-tri]"}
\clisting{lines-range}{"UnstructuredGrid2D", "std::cout"}
Основная работа происходит в классе \cvar{TestPoissonLinearTriangleWorker}.

\subsubsection{Рабочий объект}
Класс \cvar{TestPoissonLinearTriangleWorker}
наследуется от \cvar{ITestPoisson2FemWorker}. В этом классе
сформулированы двумерные аналитические функции, служащие
правой частью, точным решением и условиями первого рода уравнения Пуассона.
А этот класс в свою очередь наследуется от \cvar{ITestPoissonFemWorker},
в котором и происходит решение уравнения.
\clisting{to-start}{}
\clisting{block}{"double ITestPoissonFemWorker::solve()"}
Для получения решения сначала собирается левая и правая
часть системы линеных уравнений,
потом происходит учёт граничных условий первого рода
, вызывается решатель системы уравнений и вычислитель нормы ошибки.

Функция сборки матрицы левой части реализует сборку глобальной матрицы жёсткости
через набор локальных матриц
\clisting{to-start}{}
\clisting{block}{"CsrMatrix ITestPoissonFemWorker::approximate_lhs() const"}
Основой для сборки служит специальный объёкт \cvar{_fem} класса
\cvar{FemAssembler} -- сборщик.
Этот объёкт сначала используется для задания шаблона итоговой матрицы,
потом в цикле по элементам вычисляются локальные матрицы и с
помощью метода этого класса \cvar{FemAssembler::add_to_global_matrix}
локальные матрицы добавляются в глобальную.

По аналогичной процедуре работает и сборка правой части \cvar{approximate_rhs}.

\subsubsection{Конечноэлементный сборщик}
Конечноэлементный сборщик \cvar{FemAssembler} -- основной класс, хранящий
всю информацию о текущей конечноэлементной аппроксимации: 
массив конечных элементов и их связность.
Эта информация подаётся ему при конструировании (реализация в файле \ename{cfd24/fem/fem_assembler.hpp}).
\clisting{open}{"cfd24/fem/fem_assembler.hpp"}
\clisting{lines-range}{"FemAssembler(", ");"}
Связность \cvar{tab_elem_basis} имеет формат \quo{элемент-глобальный базис} и
определяет глобальный индекс для каждого локального базисного индекса.
В рассматренных нами узловых конечных элементах базис связан с узлом сетки.
То есть эта таблица -- это связность локальной и глобальной нумерации узлов сетки для каждой ячейки сетки.

Конечноэлементный сборщик создаётся в методе
\cvar{TestPoissonLinearTriangleWorker::build_fem} итогового
рабочего класса (то есть сборщик специфичен для конкретной сетки
и конкретного выбора типов элементов). Далее он пробрасывается в конструктор базового рабочего класса.

\subsubsection{Концепция конечного элемента}
\clisting{open}{"cfd24/fem/fem_element.hpp"}
Класс конечного элемента \cvar{FemElement} определён в файле
\ename{fem/fem_element.hpp} как
\clisting{block}{"struct FemElement"}
Главная задача объекта этого класса -- вычисление элементных матриц,
которые впоследствии используются сборщиком
для создания глобальных матриц.
Для расчёта элементных матриц в свою очередь требуется
\begin{itemize}
\item Геометрия элемента, включающая в себя правило отображения элемента из физической в параметрическую область,
\item Набор локальных базисных функций, заданных в параметрическом пространстве на указанной геометрии,
\item Непосредственно правило интегрирования в параметрической области.
\end{itemize}
Каждый из этих трёх алгоритмов определён через интерфейсы
\begin{itemize}
\item \cvar{IElementGeometry}
\item \cvar{IElementBasis}
\item \cvar{IElementIntegrals}
\end{itemize}
Определение конечного элемента заключается в задании конкретных реализаций этих интерфейсов.

\subsubsubsection{Определение линейного треугольного элемента}
\label{sec:linear_triangle_assembly}.
Так, в рассматриваемом нами тесте \ename{"[poisson2-fem-tri]"}, используются только линейные треугольные элементы.
Используется следующее определение элемента:
\clisting{open}{"test/poisson_fem_solve_test.cpp"}
\clisting{pass}{"TestPoissonLinearTriangleWorker::build"}
\clisting{lines-range}{"geom =", "FemElement"}
Здесь последовательно определяются:
\begin{itemize}
\item
треугольная геометрия \cvar{geom} -- путём задания трёх точек в физичекой плоскости \cvar{p0, p1, p2},
\item
линейный треугольный базис \cvar{basis},
\item
правила интегрирования \cvar{integrals} по параметрическому треугольнику
с использованием точных формул. Эти формулы зависят только от матрицы Якоби \cvar{jac}, которая
вычисляется с использованием геометрических свойств элемента
(в данном случае матрица Якоби постоянная, поэтому её можно вычислять в любой точке параметрической плоскости)
\end{itemize}
Из этих трёх алгоритмов собирается конечный элемент \cvar{elem}.

\subsubsubsection{Определение линейного элемента на отрезке}
В тесте, который решает аналогичную задачу на одномерных линейных элементах \ename{"[poisson1-fem-segm]"}, конечный элемент
собирается из процедур, определённых для сегмента:
\clisting{to-start}{}
\clisting{pass}{"TestPoissonLinearSegmentWorker::build"}
\clisting{lines-range}{"geom =", "FemElement"}

\subsubsubsection{Определение билинейного элемента на четырёхугольнике}
\label{sec:linear_quadrangle_assembly}.
В тесте, использующим четырёхугольные элементы \ename{"[poisson2-fem-quad]"}
определение конечного элемента имеет вид
\clisting{to-start}{}
\clisting{pass}{"TestPoissonBilinearQuadrangleWorker::build"}
\clisting{lines-range}{"geom =", "FemElement"}
Задание геометрии и базиса здесь аналогично ранее рассмотреным элементам.
А для интегрирования \cvar{integrals} используется квадратурная формула.
Расчёт интеграла по квадратурной формуле определяется классом
\cvar{NumericElementIntegrals}.
Для его определения нужно задать непосредственно квадратурную формулу
\cvar{quadrature} (здесь используется формула, точная для полиномов второго порядка, заданных на параметрическом квадртате)
а также геометрию и базис элемента. Последние нужны для вычисления подинтегральных выражений.

\subsubsubsection{Геометрические свойства элемента}
\clisting{open}{"cfd24/fem/fem_element.hpp"}
Интерфейс \cvar{IElementGeometry}, заданный в файле \ename{cfd24/fem/fem_element.hpp},
определяет геометрические свойства элемента:
\clisting{block}{"class IElementGeometry"}
Для вычиселния элементных матриц главным геометрическим свойством
элемента является функция для вычисления матрицы Якоби (\cvar{jacobi}).

Рассмотрим реализацию этого класса для линейного треугольного элемента
(в файле \ename{cfd24/fem/elem2d/triangle_linear.hpp})
\clisting{open}{"cfd24/fem/elem2d/triangle_linear.hpp"}
\clisting{block}{"class TriangleLinearGeometry"}

Для конструирования геометрии необходимо
задать три точки, определяющие треугольник в физическом пространстве.
Матрица Якоби в этом случае является постоянной
и вычисляется один раз в конструкторе по формуле \cref{eq:lintri_jacobi_matrix} с использованием переданных в конструктор точек.
Хранится вычисленная матрица в приватном поле \cvar{_jac}.

\subsubsubsection{Элементный базис}
Интерфейс для определения локального элементного базиса имеет вид
\clisting{open}{"cfd24/fem/fem_element.hpp"}
\clisting{block}{"class IElementBasis"}
Этот интерфейс работает только с параметрическим пространстсвом
и определяет следующие методы:
\begin{itemize}
\item \cvar{size} -- количество базисных функций;
\item \cvar{parametric_reference_points} -- вектор из параметрических коордианат
      точек, приписанных к соответствующим базисам;
\item \cvar{value} -- значение базисных функций в заданной точке;
\item \cvar{grad} -- градиент (в параметрическом пространстве) базисных функций по заданным точкам.
\end{itemize}

Конкретная реализация для линейного треугольного элемента \cvar{TriangleLinearBasis}
(в файле \ename{cfd24/fem/elem2d/triangle_linear.cpp})
включает в себя линейный Лагранжев базис в двумерном пространстве согласно
\cref{eq:triangle_linear_basis}:
\clisting{open}{"cfd24/fem/elem2d/triangle_linear.cpp"}
\clisting{block}{"TriangleLinearBasis::size"}
\clisting{block}{"TriangleLinearBasis::parametric_reference_points"}
\clisting{block}{"TriangleLinearBasis::value"}
\clisting{block}{"TriangleLinearBasis::grad"}


\subsubsubsection{Калькулятор элементных матриц}
Интерфейс \cvar{IElementIntegrals} предоставляет методы
для вычисления интегралов
\clisting{open}{"cfd24/fem/fem_element.hpp"}
\clisting{block}{"class IElementIntegrals"}
До сих пор были рассмотрены две элементные матрицы: матрица масс \cref{eq:mass_matrix} и матрица жёсткости \cref{eq:stiff_matrix}.
Для вычисления этих матриц используются функции \cvar{mass_matrix}, \cvar{stiff_matrix}.
Возвращают эти функции локальные квадратные матрицы с числом строк,
равным количеству базисов в элементе. 
Выходные матрицы развёрнуты в линейный массив. Так, для треугольного элемента
с тремя базисными функциями на выходе будет массив из девяти элементов:
$$
m_{00}, m_{01}, m_{02}, m_{10}, m_{11}, m_{12}, m_{20}, m_{21}, m_{22}.
$$

Два подхода к вычислению элементных интегралов: точное и численное интегрирование,
отражены в разных реализациях этого интерфейса.

Точные формулы интегрирования зависят
от вида элемента. Так, для линейного треугольного элемента
аналитическое интегрирование реализовано в классе
\cvar{TriangleLinearIntegrals} (в файле \ename{cfd24/fem/elem2d/triangle_linear.hpp}.
Все интегралы будут зависеть только от матрицы Якоби, которая и передётся этому классу в конструктор.
Например, вычисление матрицы масс (по \cref{eq:mass_matrix_lintri}) запрограммировано в виде
\clisting{open}{"cfd24/fem/elem2d/triangle_linear.cpp"}
\clisting{block}{"mass_matrix()"}

Для численного интегрирования по квадратурным формулам вида \cref{eq:quadrature_formula}
реализация этого интерфейса \cvar{NumericElementIntegrals} находится в файле
\ename{cfd24/fem/fem_numeric_integrals.hpp}.
Для вычисления локальных матриц по квадратурной формуле необходимо
знать квадратурные узлы и веса, а также
уметь вычислять подинтегральное выражения.
Конструктор этого класса имеет сигнатуру
\clisting{open}{"cfd24/fem/fem_numeric_integrals.hpp"}
\clisting{lines-range}{"NumericElementIntegrals(", ");"}
Первый аргумент -- это конкретная квадратура (специфичная для каждой геометрии элемента),
два других аргумента описывают геометрию и набор базисов элемента (эти интерфейсы были разобраны ранее).
Такой инициализации оказывается достаточно для вычисления локальных матриц
для любого конечного элемента.

\subsection{Задание для самостоятельной работы}
\begin{itemize}
\item
Определить порядок аппроксимации двумерного уравнения Пуассона на треугольных элементах.
Решение реализовано в тесте \ename{[poisson2-fem-tri]}.
Для построения треугольных сеток различного разрешения использовать
скрипт \ename{trigrid.py}.
\item
Решить уравнение Пуассона на сетке, содержащей как треугольники, так и четырёхугольники.
Для построения таких сеток использовать скрипт \ename{tetragrid.py}.
Определить порядок аппроксимации и сравнить его с решателем на треугольной сетке из предыдущего пункта.
\item
С помощью расчёта на серии сгущающихся сеток исследовать влияние точности квадратурной формулы на полученную итоговую ошибку в решении
уравнения Пуассона из предыдущего пункта (Здесь необходимо использовать квадратурные формулы
в том числе и для треугольных элементов, где интегралы до сих пор вычислялись аналитически).
\end{itemize}

Для решения уравнения Пуассона {\bf на сетке треугольников и четырёхугольников} необходимо создать новый тест \ename{"poisson2-fem-tri-quad"}:
\begin{cppcode}
TEST_CASE("Poisson-fem 2D solver, triangles & quadrangles", "[poisson2-fem-tri-quad]")
\end{cppcode}
и новый рабочий класс
\begin{cppcode}
struct TestPoissonLinearTriQuadWorker: public ITestPoisson2FemWorker{
	static FemAssembler build_fem(const IGrid& grid);
	TestPoissonLinearTriQuadWorker(const IGrid& grid):
		ITestPoisson2FemWorker(grid, build_fem(grid)){ }
};
\end{cppcode}
Учёт наличия разных элементов следует учитывать при реализации функции \cvar{build_fem}.
Если в ячейке три узла, то следует собирать треугольный элемент по
алгоритму, описанному в п.~\ref{sec:linear_triangle_assembly},
а если четыре, то согласно п.~\ref{sec:linear_quadrangle_assembly}.

Для задания правил {\bf интегрирования с использованием квадратурных формул разного порядка точности} следует
использовать класс \cvar{NumericElementIntegrals} в качестве реализации
интерфейса \cvar{IElementIntegrals}. Пример его использования
указан в листинге к п.~\ref{sec:linear_quadrangle_assembly}.
Реализованные в программе квадратуры:
\begin{itemize}
\item Квадратурные формулы для параметрического треугольника, точные для полиномов первой, второй, третьей и четвёртой степеней.
(то есть имеющие второй, третий, четвёртный и пятый порядки аппроксимации):
\begin{cppcode}
quadrature_triangle_gauss1();
quadrature_triangle_gauss2();
quadrature_triangle_gauss3();
quadrature_triangle_gauss4();
\end{cppcode}
\item Квадратурные формулы для параметрического квадрата
\begin{cppcode}
quadrature_square_gauss1();
quadrature_square_gauss2();
quadrature_square_gauss3();
quadrature_square_gauss4();
\end{cppcode}
\end{itemize}
