\section{Лекция 11 (25.11)}
\subsection{Метод взвешенных невязок}
\subsection{Метод Бубнова--Галёркина}
\subsubsection{Степенные базисные функции}
\subsection{Метод конечных элементов}
\subsubsection{Узловые базисные функции}
\subsubsection{Одномерное уравнение Пуассона}
\subsubsubsection{Слабая интегральная постановка задачи}
\subsubsubsection{Линейный одномерный базис}
\subsubsubsection{Элементные матрицы}
Матрица масс
\begin{equation}
\label{eq:mass_matrix}
M^E_{ij} = \arint{\phi_i(x) \phi_j(\vec x)}{E}{\vec x} = \arint{\phi_i(\vec \xi) \phi_j(\vec \xi) |J(\vec \xi)|}{E^p}{\vec \xi}
\end{equation}
Вектор нагрузок
\begin{equation}
\label{eq:load_vector}
L^E_{i} = \arint{\phi_i(x)}{E}{\vec x} = \arint{\phi_i(\vec \xi) |J(\vec \xi)|}{E^p}{\vec \xi}
\end{equation}
Матрица жёсткости
\begin{equation}
\label{eq:stiff_matrix}
S^E_{ij} = \arint{\nabla \phi_i \cdot \nabla \phi_j}{E}{\vec x} = \arint{\nabla \phi_i \cdot \nabla \phi_j |J(\vec \xi)|}{E^p}{\vec \xi}
\end{equation}
\subsubsubsection{Сборка глобальных матриц и векторов}
\subsubsection{Двумерное уравнение Пуассона}
\subsubsubsection{Треугольный элемент. Линейный двумерный базис}
Матрица масс (из \cref{eq:mass_matrix}):
\begin{equation}
\label{eq:mass_matrix_lintri}
M^E_{ij} = \int\limits_0^1 \int\limits_0^{1-\xi} \phi_i(\xi, \eta) \phi_j(\xi, \eta) |J| \, d\eta d\xi =
\frac{|J|}{24}\left(
\begin{array}{ccc}
2 & 1 & 1 \\
1 & 2 & 1 \\
1 & 1 & 2
\end{array}
\right)
\end{equation}

\subsubsubsection{Пример сборки матрицы для двумерной задачи}
