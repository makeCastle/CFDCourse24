\subsection{Git}
\subsubsection{Основные команды}
Все команды выполнять в терминале (\ename{git bash} для виндоус),
находясь в корневой папке проета CFDCourse24.
\begin{itemize}
\item
  Для {\bf смены директории} использовать команду \ename{cd}. Например, находясь в папке \ename{A} перейти в папку \ename{A/B/C}
  \begin{shelloutput}
> cd B/C
  \end{shelloutput}
\item
  {\bf Подняться} на директорию выше
  \begin{shelloutput}
> cd ..
  \end{shelloutput}
\item
  {\bf Просмотр статуса} текущего репозитория: текущую ветку, все изменённые файлы и т.п.
  \begin{shelloutput}
> git status
  \end{shelloutput}
\item
  {\bf Сохранить и скоммитить} изменения в текущую ветку
  \begin{shelloutput}
> git add .
> git commit -m "message"
  \end{shelloutput}

  ``message'' -- произвольная информация о текущем коммите, которая будет приписана к этому коммиту
\item
  {\bf Переключиться на ветку} main
  \begin{shelloutput}
> git checkout main
  \end{shelloutput}

  работает только в том случае, если все файлы скоммичены и статус ветки 'Up to date'
\item
  {\bf Создать новую ветку} ответвлённую от последнего коммита текущей ветки и переключиться на неё
  \begin{shelloutput}
> git checkout -b new-branch-name
  \end{shelloutput}

  new-branch-name -- имя новой ветки. Пробелы не допускаются

  Эта комманда работает даже если есть нескоммиченные изменения. 
  Если необходимо скоммитить изменеия в новую ветку, сразу за этой командой нужно вызвать
  \begin{shelloutput}
> git add .
> git commit -m "message"
  \end{shelloutput}
\item
  {\bf Сбросить} все нескоммиченные изменения. Вернуть файлы в состояние последнего коммита
  \begin{shelloutput}
> git reset --hard
  \end{shelloutput}

  Все изменения будут утеряны
\item
  {\bf Получить последние изменения} из удалённого хранилища с обновлением текущей ветки
  \begin{shelloutput}
> git pull
  \end{shelloutput}
  Работает только если статус текущей ветки 'Up to date'.\\
  Если требуется получить изменения, но не обновлять локальную ветку:
  \begin{shelloutput}
> git fetch
  \end{shelloutput}
  Обновленная ветка будет доступна по имени origin/{имя ветки}.
\item
  {\bf Просмотр истории} коммитов в текущей ветке (последний коммит будет наверху)
  \begin{shelloutput}
> git log
  \end{shelloutput}
\item
  {\bf Просмотр доступных веток} в текущем репозитории
  \begin{shelloutput}
> git branch
  \end{shelloutput}
\item
  {\bf Просмотр} актуального состояние дерева репозитория в gui режиме
  \begin{shelloutput}
> git gui
  \end{shelloutput}
  Далее в меню \ename{Repository->Visualize all branch history}.
  В этом же окне можно посмотреть изменения файлов по сравнению с последним коммитом.

  Альтернативно, при работе в виндоус можно установить программу GitExtensions и работать в ней.
\end{itemize}
  
\subsubsection{Порядок работы с репозиторием CFDCourse}

Основная ветка проекта -- \ename{main}. После каждой лекции (в течении 1-2 дней) в эту ветку будет отправлен коммит с сообщением \ename{after-lect{index}}.
Этот коммит будет содержать краткое содержание лекции,
задание по итогам лекции и необходимые для этого задания изменения кода.

Если предполагается работа с кодом на лекции, то перед лекцией в эту ветку будет отправлен коммит с сообщением \ename{before-lect{index}}.
Этот коммит содержит изменения кода для работы на лекции.

Таким образом, {\bf после лекции} необходимо выполнить следующие команды (находясь в ветке \ename{main})
\begin{shelloutput}
> git reset --hard  # очистить локальную копию от изменений,
                    # сделанных на лекции (если они не представляют ценности)
> git pull          # получить изменения
\end{shelloutput}

{\bf Перед началом лекции}, если была сделана какая то работа по заданиям,
\begin{shelloutput}
> git checkout -b work-lect{index}    # создать локальную ветку, содержащую задание
> git add .
> git commit -m "{свой комментарий}"  # скоммитить свои изменения в эту ветку
> git checkout main                   # вернуться на ветку main
> git pull                            # получить изменения
\end{shelloutput}

Даже если задание выполнено не до конца, вы в любой момент можете переключиться на ветку с заданием и его доделать
\begin{shelloutput}
> git checkout work-lect{index}
\end{shelloutput}

Если ничего не было сделано (или все изменения не представляют ценности), можно повторить алгоритм ``после лекции''.
