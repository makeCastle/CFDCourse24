\section{Лекция 13 (09.12)}

\subsection{Узловые элементы высокого порядка точности}
\subsection{Эрмитовы элементы}
\subsection{Разбор программной реализации МКЭ третьего порядка}
\subsection{Задание для самостоятельной работы}
Провести сравнительный анализ на порядок аппроксимации решения двумерного
уравнения Пуассона с граничными условиями первого рода
следующих конечноэлементных схем
\begin{enumerate}
\item
Линейные треугольные элементы (\figref{fig:triangle_basis_points}а)
\item
Квадратричные треугольные элементы (\figref{fig:triangle_basis_points}б)
\item
Кубические треугольные элементы (\figref{fig:triangle_basis_points}в)
\item
Квадратичные четырехугольные элементы (\figref{fig:quadrangle_basis_points}б)
\item
Неполные 8-узловые квадратичные четырёхугольные элементы (\figref{fig:quadrangle_basis_points}г). В качестве базисных функций использовать полиномы вида
$$
P(\xi, \eta) =
  A^{(00)}
+ A^{(10)}\xi
+ A^{(01)}\eta
+ A^{(11)}\xi\eta
+ A^{(20)} \xi^2
+ A^{(02)}\eta^2
+ A^{(21)}\xi^2\eta
+ A^{(12)}\xi\eta^2
+ \xcancel{\color{gray} A^{(22)}\xi^2 \eta^2}
$$
\item
Неполные 9-узловые кубические треугольные элементы (\figref{fig:triangle_basis_points}г). В качестве базисных функций использовать полиномы вида
$$
P(\xi, \eta) =
  A^{(00)}
+ A^{(10)}\xi
+ A^{(01)}\eta
+ \xcancel{\color{gray} A^{(11)}\xi\eta}
+ A^{(20)}\xi^2
+ A^{(02)}\eta^2
+ A^{(21)}\xi^2\eta
+ A^{(12)}\xi\eta^2
+ A^{(30)}\xi^3
+ A^{(03)}\eta^3
$$
\end{enumerate}

Все необходимые тесты находятся в файле \ename{poisson_fem_solve_test.cpp}.

Для {\bf линейных треугольных} элементов использовать тест \ename{[poisson2-fem-tri]}.

Для {\bf квадратичных треугольных} и {\bf квадратичных четырёхгольных} элементов -- \ename{[poisson2-fem-quadratic]}.

Для {\bf кубических треугольных} элементов -- \ename{[poisson2-fem-cubic]}.

Для {\bf неполных квадтратичных четырёхугольных} элементов в качестве основы взять тест \ename{poisson2-fem-quadratic}.
При этом необходимо изменить тип использованного четырёхугольного элемента: вместо
класса \cvar{QuadrangleQuadraticBasis} использовать \cvar{QuadrangleQuadratic8Basis}.
Кроме того, поскольку центральная точка четырёхугольника больше не используется, нужно
убрать последние (девятые) записи в таблицах связности \cvar{tab_elem_basis} для четырёхугольных элементов и
уменьшить общее количество базисных функций до
\begin{cppcode}
size_t n_bases = grid.n_points() + grid.n_faces();
\end{cppcode}

Для {\bf неполных кубических треугольных} элементов систему базисных функций нужно
вычислить самостоятельно используя алгоритм из п.\ref{sec:triangle_bases}.
На основе полученных соотношений нужно создать класс
\begin{cppcode}
class TriangleCubicNo11Basis: public IElementBasis{
public:
	size_t size() const override;
	std::vector<Point> parametric_reference_points() const override;
	std::vector<BasisType> basis_types() const override;
	std::vector<double> value(Point xi) const override;
	std::vector<Vector> grad(Point xi) const override;
};
\end{cppcode}
и реализовать все необходимые функции:
\begin{itemize}
\item \cvar{size} -- общее количество базисных функций. Здесь будет девять,
\item \cvar{parametric_reference_points} -- параметрические координаты девяти узловых точек (соблюдая порядок локальной индексации),
\item \cvar{basis_types} -- типы базисных функций. Здесь все базисы узловые (\cvar{BasisType::Nodal}),
\item \cvar{value} -- значение девяти базисных функций в заданной параметрической точке. Здесь нужно подставить полученные при вычислении базисы,
\item \cvar{grad} -- градиенты вычисленных базисных функций. Для заполнения нужно подсчитать аналитические производные по $\xi$ и $\eta$ вычислиенных базисов.
\end{itemize}
Реализовать этот класс можно взяв в качестве основы уже реализованный класс \cvar{TrinagleCubic9Basis} из файла \ename{cfd24/fem/elem2d/traingle_cubic.hpp}.
После того, как базис будет реализован, его нужно использовать в тесте \ename{poisson2-fem-cubic}:
\begin{cppcode}
auto basis = std::make_shared<TriangleCubicNo11Basis>();
\end{cppcode}
Аналогично ранее рассмотренному неполному квадратичному элементу,
следует убрать последний базис из таблицы связности
\begin{cppcode}
tab_elem_basis.push_back({
	bas0, bas1, bas2,
	bas3, bas4, bas5, bas6, bas7, bas8
});
\end{cppcode}
и сократить общее количество базисных функций
\begin{cppcode}
size_t n_bases = grid.n_points() + 2*grid.n_faces();
\end{cppcode}
