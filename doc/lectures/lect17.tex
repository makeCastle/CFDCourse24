\section{Лекция 17 (30.03)}

\subsection{Метод конечных элементов со стабилизацией}
\begin{equation}
\label{eq:fem_supg_g1}
\mat G_1 = \feint{\nabla^2 \phi_j \left(\vec v \cdot \nabla \phi_i\right)},
\end{equation}

\subsection{Задание для самостоятельной работы}
\label{sec:hw_supg}
В тестовом примере \cvar{[convdiff-fem-supg]}
из файла \ename{convdiff-fem-test.cpp}
производится численного решение одномерного нестационарного уравнения конвекции-диффузии
$$
\dfr{u}{t} + \dfr{u}{x} - \eps \dfrq{u}{x} = 0
$$
в области $x\in[0, 4]$
с точным решением вида
$$
u^e(x, t) = \frac{1}{\sqrt{4\pi \eps (t + t_0)}} \exp\left(-\frac{(x - v_x t)^2}{4\eps(t+t_0)}\right)
$$
Точное решение используется для формулировки начальных ($t=0$) и граничных ($x=0,4$) условий первого рода.
Результат расчёта сохраняется в файл \ename{convdiff-supg.vtk.series}.

Задача полудискретизуется по схеме Кранка-Николсон ($\theta=\sfrac12$) c шагом по времени, вычисленным через число Куранта $\rm Cu = 0.5$.

После дискретизации задача сводится к СЛАУ относительно неизвестного сеточного вектора $u$
$$
(A + s A^s) u = (B + s B^s) \check u. \\
$$
Матрицы расчитываются по процедуре Бубнова-Галёркина и SUPG стабилизаторов (матрицы с индексом $^s$).
Множитель $s$ -- параметр SUPG-стабилизации.
$$
\begin{array}{ll}
A = M + \tau\theta K + \eps\tau\theta S, &
B = M - \tau(1-\theta)K - \eps\tau(1-\theta) S, \\
A^s = M^s + \tau\theta K^s + \eps\tau\theta S^s, &
B^s = M^s - \tau(1-\theta)K^s - \eps\tau (1-\theta) S^s, \\
\end{array}
$$

Стабилизирующие матрицы вычислялись по следующим формулам:
$$
M^s = \int\limits_\Omega \phi_j \vec v \cdot \nabla \phi_i \, d\vec x
$$
$$
K^s = \int\limits_\Omega \vec v \cdot \nabla \phi_j \vec v \cdot \nabla \phi_i \, d\vec x
$$
$$
S^s = \int\limits_\Omega \nabla^2 \phi_j \vec v \cdot \nabla \phi_i \, d\vec x
$$
Последний интеграл из-за использования линейных базисов был равен нулю.

Сборки необходимых матриц осуществляется в процедуре \cvar{asseble_solver}.
После её окончания производится учёт граничных условий первого рода на матричном уровне.


Необходимо:
\begin{enumerate}
\item В одномерном тесте \cvar{[convdiff-supg]} с помощью анимированных графиков продемонстрировать наличие осцилляций
      при выбранных параметрах решения при отсутствии стабилизации ($s = 0$), а так же эффект от SUPG-слагаемого ($s > 0$);
\item Нарисовать в сравнении результаты расчёта на конечный промежуток времени для различных $s$;
\item Подобрать оптимальную величину параметра $s$, минимизирующую норму отклонения численного решения от точного;
\item Написать аналогичный тест для двумерного случая (решение в единичном квадрате). Точное решение в этом случае будет иметь вид
$$
u^e(x, t) = \frac{1}{4\pi \eps (t + t_0)} \exp\left(-\frac{(x - v_x t)^2 + (y - v_y t)^2}{4\eps(t+t_0)}\right)
$$
Использовать $v_x = v_y = 1$.
\item Проиллюстрировать работу программы на cкошенной и структурированной двумерных сетках.
\end{enumerate}

При программировании двумерного решателя нужно 
\begin{itemize}
\item изменить значение точного решения и скорости (функции \cvar{velocity}, \cvar{nonstat_solution});
\item внести изменения в процедуру постановки граничных условий в функциях \cvar{assemble_solver}, \cvar{assemble_rhs} (сместо двух точек начала и конца одномерной области
необходимо пройтись по всем граничным узлам двумерной сетки);
\end{itemize}
