\section{Лекция 9 (11.11)}

\subsection{Решение системы Уравнений Навье-Стокса методом конечных объёмов}
TODO

\subsubsection{Схема SIMPLE}
TODO

\subsubsection{Вычисление градиента давления методом наименьших квадратов}
TODO

\subsubsection{Интерполяция Rhie-Chow нормальной компоненты скорости} 
TODO

\subsubsection{Порядок вычисления на итерации}
TODO

\subsection{Пример расчётной программы. Течение в каверне}
Представлена в файле \ename{cavern_2d_fvm_simple_test.cpp}
в тесте \ename{[cavern2-fvm-simple]}

TODO

\subsection{Задание для самостоятельной работы}
На основе теста \ename{[cavern2-fvm-simple]} из файла \ename{cavern_2d_fvm_simple_test.cpp}
сравнить результат решения задачи
о течении в каверне на
структурированной, pebi и произвольной сетках.
Использовать количество элементов $N\approx2000$.
Построение и чтение сеток проводить по аналогии с п.\ref{seq:fvm_poisson_hw}.

\begin{enumerate}
\item
Нарисовать результат решения (давление и векторы скорости)
на различных итерациях для всех использованных сеток.
Отметить количество требуемых итераций до сходимости с $\eps=10^{-2}$.

\item
На основе полученного решения построить и сохранить
в выходной vtk файл дивергенцию скорости и невязку решения.

\item
Построить графики сходимости невязки (печатается в консоль при итерациях) от номера итерации для трёх сеток.
\end{enumerate}

\paragraph{Вычисление дивиргенции скорости}
Распишем значения дивергенции скорости
в ячейке $E_i$ как среднеинтегральное и далее воспользуемся формулой Гаусса-Остроградского \cref{eq:partint_div}:
$$
\left(\nabla \cdot \vec u\right)_i \approx
\frac{1}{|E_i|}\arint{\nabla \cdot \vec u}{E_i}{\vec x} =
\frac{1}{|E_i|}\sum_j \arint{u_n}{\gamma_{ij}}{s} \approx
\frac{1}{|E_i|}\sum_j \left(u_n\right)_{ij} |\gamma_{ij}|,
$$
где $\gamma_{ij}$ -- все грани ячейки $E_i$, а $(u_n)_{ij}$ -- значение нормальной (внешней нормали по отношению к ячейке $E_i$) скорости
на этой грани.
Расчёт по этой формуле нужно вести в цикле по граням, определяя
для каждой грани пару соседних ячеек.
\begin{equation*}
\begin{array}{ll}
d = \{0, ...\}                                       & \textrm{-- массив дивергенций. Длина равна количеству ячеек} \\
\textbf{for } s=\overline{0, N_f-1}                  & \textrm{-- цикл по всем граням} \\
\qquad i, j = \textrm{nei\_cells(s)}                 & \textrm{-- ячейки, соседние с гранью: против и по нормали} \\
\qquad c = (u_n)_s |\gamma_s|                        & \\
\qquad \textbf{if } (i \neq \textrm{INVALID\_INDEX}) & \textrm{-- если слева от грани есть ячейка}\\
\qquad \qquad d_i \pluseq  \sfrac{c}{|E_i|}          & \textrm{-- добавляем, так как нормаль внешняя для ячейки i}\\
\qquad \textbf{endif}                                & \\
\qquad \textbf{if } (j \neq \textrm{INVALID\_INDEX}) & \textrm{-- если справа от грани есть ячейка}\\
\qquad \qquad d_j \minuseq \sfrac{c}{|E_j|}          & \textrm{-- вычитаем, так как нормаль внутренняя для ячейки j} \\
\qquad \textbf{endif}                                & \\
\textbf{endfor}
\end{array}
\end{equation*}
Массив нормальных скоростей к граням сетки доступен как поле рабочего класса \cvar{Cavern2DFvmSimpleWorker::_un_face}.
Методы сетки \cvar{IGrid}, необходимые для программирования этой формулы:
\begin{itemize}
\item \cvar{IGrid::n_cells()}, \cvar{IGrid::n_faces()} -- количество ячеек и граней сетки,
\item \cvar{IGrid::tab_face_cell(iface)} -- индексы пары соседних с гранью ячеек. Первая -- против нормали, вторая -- по.
Для граничных граней один из возвращаемых индексов равен \cvar{INVALID_INDEX}.
\item \cvar{IGrid::cell_volume(icell)} -- объём ячейки $E_i$,
\item \cvar{IGrid::face_area(iface)} -- площадь грани $\gamma_s$.
\end{itemize}
Вычисление массива дивиргенций можно производить
непосредственно в процедуре перед сохраненим
решения
\cvar{Cavern2DFvmSimpleWorker::save_current_fields} с тем, чтобы
сразу сохранить полученный массив дивергениций в файл вызовом
\cvar{VtkUtils::add_cell_data(d, "div", filepath);}


\paragraph{Вычисление массива невязок}
происходит в функции установки текущих значение решения \cvar{set_uvp}:
\clisting{open}{"test/cavern_2d_fvm_simple_test.cpp"}
\clisting{block}{"std::vector<double> res_u = "}

Для его сохранения в файл нужно либо повторить эту процедуру в функции сохранения,
а лучше сделать \cvar{rhs_u}, \cvar{rhs_v} полями рабочего класса,
чтобы иметь к ним доступ в методах класса.
Для их сохранения нужно в процедуре сохранения \cvar{save_current_fields}
вызвать
\begin{cppcode}
VtkUtils::add_cell_data(res_u, "res_u", filepath);
VtkUtils::add_cell_data(res_v, "res_v", filepath);
\end{cppcode}


